For the ensemble method we combined the interaction probabilities of the other methods in the following way: each method generates a vector of interaction probabilities. Those probabilities can also be interpreted as rankings. To combine the rankings we gave weights to each rank. The lowest rank got a weight of zero, the second lowest rank a weight of one, the third lowest rank a weight of two and so forth. To calculate the combined ranks we simply added up the weights given by each method for a particular item. For example, the item with the most added up weight would now have the highest combined rank. Then the combined ranks can now again be interpreted as interaction probabilities.

We choose this method because of it's simplicity and empirically it already delivers interesting results, however there are many ways how rankings or probability vectors can be combined and other methods might yield even better results.
%%% Local Variables:
%%% mode: latex
%%% TeX-master: "../../report"
%%% End:
