\textbf{TODO: }
\textit{Here we will present the tables and plots of the comparative results.}
\textit{Also state the methods' codenames. For example, Generalized Matrix Factorization with embedding size 8 has a codename GMF(8), Multilayer Perceptron with two layers will be called MLP(2), Neural Matrix Factorization with embedding size 8 and 2 layers will be called NeuMF(8,2) ... Something like that.}
\begin{table}[h]
    \centering
    \begin{tabular}{c|c|c}
        \hline
        Movielens & HR@10  & NDCG@10 \\
        \hline
        GMF(32)     & 0.7023 & 0.4202  \\
        MLP(4)      & 0.6616 & 0.3897 \\
        NeuMF(32,2) & 0.7012 & 0.4233 \\
        CMN         & 0 & 0 \\
        NGCF        & 0 & 0 \\
        VAE         & 0 & 0
    \end{tabular}
    \caption{Best performance achieved by each method (state configurations on codename) on Movielens dataset.}
    \label{tab:movielens}
\end{table}

\begin{table}[h]
    \centering
    \begin{tabular}{c|c|c}
        \hline
        Jester & HR@10 & NDCG@10 \\
        \hline
        GMF(16)    & 0.8411 & 0.7589 \\
        MLP(3)     & 0.8427 & 0.7569 \\
        NeuMF(8,2) & 0.8404 & 0.7563 \\
        CMN        & 0 & 0 \\
        NGCF       & 0 & 0 \\
        VAE        & 0 & 0
    \end{tabular}
    \caption{Best performance achieved by each method (state configurations on codename) on Jester dataset.}
    \label{tab:jester}
\end{table}

\begin{table}[h]
    \centering
    \begin{tabular}{c|c|c}
        \hline
        Epinions & HR@10 & NDCG@10 \\
        \hline
        GMF(64)     & 0.2348 & 0.1351 \\
        MLP(2)      & 0.3641 & 0.2135 \\
        NeuMF(32,2) & 0.3750 & 0.2167 \\
        CMN         & 0 & 0 \\
        NGCF        & 0 & 0 \\
        VAE         & 0 & 0
    \end{tabular}
    \caption{Best performance achieved by each method (state configurations on codename) on Epinions dataset.}
    \label{tab:epinions}
\end{table}

% Plot HR vs k, for best tuned models and for k=1,...,10, for all datasets
% \pgfplotstabletypeset{data.dat}

% Note about .dat file formating.
% Should have 3 columns, i.e.:
% k HR NDCG
% The k is ranging from 1 to 10.
% The respective Hit Ratios under HR.
% The respective NDCGs under NDCG.
\begin{figure*}[t]
    \centering
    %%%%%%%%%%%%%%%%%%%%%%%%%%%%%%%%%%%
    %%%%%% Hit Ratio - Movielens %%%%%%
    %%%%%%%%%%%%%%%%%%%%%%%%%%%%%%%%%%%
    \begin{subfigure}[ht]{0.3\textwidth}
        \centering
        \begin{tikzpicture}[scale=0.5]
        \begin{axis}[xlabel=k,ylabel=HR@k, legend pos=south east]
        \addplot table [x=k, y=HR]{results/movielens_gmf.dat};
        \addlegendentry{GMF(32)}
        \addplot table [x=k, y=HR]{results/movielens_mlp.dat};
        \addlegendentry{MLP(4)}
        \addplot table [x=k, y=HR]{results/movielens_neumf.dat};
        \addlegendentry{NeuMF(32,2)}
        \addplot table [x=k, y=HR]{results/movielens_ngcf.dat};
        \addlegendentry{NGCF(64)}
        \end{axis}
        \end{tikzpicture}
        \caption{Movielens}
    \end{subfigure}
    %%%%%%%%%%%%%%%%%%%%%%%%%%%%%%%%%%%
    %%%%%%% Hit Ratio - Jester %%%%%%%%
    %%%%%%%%%%%%%%%%%%%%%%%%%%%%%%%%%%%
    \begin{subfigure}[ht]{0.3\textwidth}
        \centering
        \begin{tikzpicture}[scale=0.5]
        \begin{axis}[xlabel=k,ylabel=HR@k, legend pos=south east]
        \addplot table [x=k, y=HR]{results/jester_gmf.dat};
        \addlegendentry{GMF(16)}
        \addplot table [x=k, y=HR]{results/jester_mlp.dat};
        \addlegendentry{MLP(3)}
        \addplot table [x=k, y=HR]{results/jester_neumf.dat};
        \addlegendentry{NeuMF(8,2)}
        \addplot table [x=k, y=HR]{results/jester_ngcf.dat};
        \addlegendentry{NGCF(32)}
        \end{axis}
        \end{tikzpicture}
        \caption{Jester}
    \end{subfigure}
    %%%%%%%%%%%%%%%%%%%%%%%%%%%%%%%%%%%%
    %%%%% Hit Ratio - Epinions %%%%%%%%%
    %%%%%%%%%%%%%%%%%%%%%%%%%%%%%%%%%%%%
    \begin{subfigure}[ht]{0.3\textwidth}
        \centering
        \begin{tikzpicture}[scale=0.5]
        \begin{axis}[xlabel=k,ylabel=HR@k, legend pos=south east]
        \addplot table [x=k, y=HR]{results/epinions_gmf.dat};
        \addlegendentry{GMF(64)}
        \addplot table [x=k, y=HR]{results/epinions_mlp.dat};
        \addlegendentry{MLP(2)}
        \addplot table [x=k, y=HR]{results/epinions_neumf.dat};
        \addlegendentry{NeuMF(32,2)}
        \addplot table [x=k, y=HR]{results/epinions_ngcf.dat};
        \addlegendentry{NGCF(8)}
        \end{axis}
        \end{tikzpicture}
        \caption{Epinions}
    \end{subfigure} 

    %%%%%%%%%%%%%%%%%%%%%%%%%%%%%%%%%%%%
    %%%%%%%%% NDCG - Movielens %%%%%%%%%
    %%%%%%%%%%%%%%%%%%%%%%%%%%%%%%%%%%%%
    \begin{subfigure}[ht]{0.3\textwidth}
        \centering
        \begin{tikzpicture}[scale=0.5]
        \begin{axis}[xlabel=k,ylabel=NDCG@k, legend pos=south east]
        \addplot table [x=k, y=NDCG]{results/movielens_gmf.dat};
        \addlegendentry{GMF(32)}
        \addplot table [x=k, y=NDCG]{results/movielens_mlp.dat};
        \addlegendentry{MLP(4)}
        \addplot table [x=k, y=NDCG]{results/movielens_neumf.dat};
        \addlegendentry{NeuMF(32,2)}
        \addplot table [x=k, y=NDCG]{results/movielens_ngcf.dat};
        \addlegendentry{NGCF(64)}
        \end{axis}
        \end{tikzpicture}
        \caption{Movielens}
    \end{subfigure}
    %%%%%%%%%%%%%%%%%%%%%%%%%%%%%%%%%%%
    %%%%%%%%%% NDCG - Jester %%%%%%%%%%
    %%%%%%%%%%%%%%%%%%%%%%%%%%%%%%%%%%%
    \begin{subfigure}[ht]{0.3\textwidth}
        \centering
        \begin{tikzpicture}[scale=0.5]
        \begin{axis}[xlabel=k,ylabel=NDCG@k, legend pos=south east]
        \addplot table [x=k, y=NDCG]{results/jester_gmf.dat};
        \addlegendentry{GMF(16)}
        \addplot table [x=k, y=NDCG]{results/jester_mlp.dat};
        \addlegendentry{MLP(3)}
        \addplot table [x=k, y=NDCG]{results/jester_neumf.dat};
        \addlegendentry{NeuMF(8,2)}
        \addplot table [x=k, y=NDCG]{results/jester_ngcf.dat};
        \addlegendentry{NGCF(32)}
        \end{axis}
        \end{tikzpicture}
        \caption{Jester}
    \end{subfigure}
    %%%%%%%%%%%%%%%%%%%%%%%%%%%%%%%%%%%%
    %%%%%%%%% NDCG - Epinions %%%%%%%%%%
    %%%%%%%%%%%%%%%%%%%%%%%%%%%%%%%%%%%%
    \begin{subfigure}[ht]{0.3\textwidth}
        \centering
        \begin{tikzpicture}[scale=0.5]
        \begin{axis}[xlabel=k,ylabel=NDCG@k, legend pos=south east]
        \addplot table [x=k, y=NDCG]{results/epinions_gmf.dat};
        \addlegendentry{GMF(64)}
        \addplot table [x=k, y=NDCG]{results/epinions_mlp.dat};
        \addlegendentry{MLP(2)}
        \addplot table [x=k, y=NDCG]{results/epinions_neumf.dat};
        \addlegendentry{NeuMF(32,2)}
        \addplot table [x=k, y=NDCG]{results/epinions_ngcf.dat};
        \addlegendentry{NGCF(8)}
        \end{axis}
        \end{tikzpicture}
        \caption{Epinions}
    \end{subfigure} 
    \caption{Evaluation of top k item recommendation, where k ranges from 1 to 10 on the three datasets.}
\end{figure*}

%%% Local Variables:
%%% mode: latex
%%% TeX-master: "../../report"
%%% End:


%%% Local Variables:
%%% mode: latex
%%% TeX-master: "../report"
%%% End:
