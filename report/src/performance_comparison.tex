\textbf{TODO: }
\textit{Here we will present the tables and plots of the comparative results.}
\textit{Also state the methods' codenames. For example, Generalized Matrix Factorization with embedding size 8 has a codename GMF(8), Multilayer Perceptron with two layers will be called MLP(2), Neural Matrix Factorization with embedding size 8 and 2 layers will be called NeuMF(8,2) ... Something like that.}
\begin{table}[h]
    \centering
    \begin{tabular}{c|c|c}
        \hline
        Movielens & HR@10 & NDCG@10 \\
        \hline
        GMF   & 0 & 0 \\
        MLP   & 0 & 0 \\
        NeuMF & 0 & 0 \\
        CMN   & 0 & 0 \\
        NGCF  & 0 & 0 \\
        VAE   & 0 & 0
    \end{tabular}
    \caption{Best performance achieved by each method (state configurations on codename) on Movielens dataset.}
    \label{tab:movielens}
\end{table}

\begin{table}[h]
    \centering
    \begin{tabular}{c|c|c}
        \hline
        Jester & HR@10 & NDCG@10 \\
        \hline
        GMF   & 0 & 0 \\
        MLP   & 0 & 0 \\
        NeuMF & 0 & 0 \\
        CMN   & 0 & 0 \\
        NGCF  & 0 & 0 \\
        VAE   & 0 & 0
    \end{tabular}
    \caption{Best performance achieved by each method (state configurations on codename) on Jester dataset.}
    \label{tab:jester}
\end{table}

\begin{table}[h]
    \centering
    \begin{tabular}{c|c|c}
        \hline
        Epinions & HR@10 & NDCG@10 \\
        \hline
        GMF   & 0 & 0 \\
        MLP   & 0 & 0 \\
        NeuMF & 0 & 0 \\
        CMN   & 0 & 0 \\
        NGCF  & 0 & 0 \\
        VAE   & 0 & 0
    \end{tabular}
    \caption{Best performance achieved by each method (state configurations on codename) on Epinions dataset.}
    \label{tab:epinions}
\end{table}

%%% Local Variables:
%%% mode: latex
%%% TeX-master: "../report"
%%% End:
