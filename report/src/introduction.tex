Recommender systems are information filtering techniques that aim to predict the level of preference of a user over a specific item. 
In the era of big data, such techniques have attracted the interest of the scientific community, as they provide a natural approach to improving the user experience on various services, through personalization.
Classical recommender systems usually make use of either content-based or collaborative filtering approaches. 
Content-based filtering techniques utilize specific characteristics of an item in order to recommend additional items with similar properties, while collaborative filtering approaches utilize users' past behaviour i.e. preferences and interactions with items, as well as decisions of other users with similar interests. 
In most cases, collaborative filtering (CF) techniques yield improved predictions compared to the content-based approaches.
There are two main categories of methods when it comes to CF; (i) the Nearest-Neighbor techniques, and (ii) the Matrix Factorization (aka Latent Factor) methods.
As the Netflix Prize competition has demonstrated, Matrix Factorization methods are superior to classic Nearest-Neighbor techniques, as they allow the incorporation of additional information to the models, and can thus achieve improved model capacity \cite{koren2009matrix}.

Recently, both academia and industry have been in a race to design deep learning based recommender systems in an attempt to overcome the obstacles of conventional models and to achieve higher recommendation quality. 
In fact, deep learning can effectively capture non-linear and non-trivial user-item relationships, and also enable codification of more complex abstractions as data representations in the higher levels \cite{zhang2019deep}.
Various deep neural network architectures have been proposed and shown to be effective for predicting user preferences. 
Neural Collaborative Filtering (NCF) generalizes the Matrix Factorization (MF) approach by replacing the inner product utilized in MF models by a multi-layer perceptron that can learn non-linear user-item interaction functions, and thus increases the expressiveness of the MF model \cite{he2017neural}. 
Collaborative Memory Networks (CMN) unify the two classes of collaborative filtering models into a hybrid approach, combining the strengths of the global structure of the latent factor model, and the local neighborhood-based structure in a nonlinear fashion, by fusing a memory component and a neural attention mechanism as the neighborhood component \cite{ebesu2018collaborative}. 
Neural Graph Collaborative Filtering (NGCF) injects the collaborative signal into the embedding process by exploiting the user-item graph structure, so that it can effectively model high-order connectivity in the user-item interaction graph, and thus achieves improved recommendation quality \cite{wang2019neural}. 
Other deep learning based recommendation methods include Autoencoders, \cite{sedhain2015autorec}, Variational Autoencoders (VAE), \cite{liang2018variational}, and Restricted Boltzmann Machines (RBMs) \cite{salakhutdinov2007restricted}. 
However, as authors have stated in \cite{dacrema2019we} there has been a reproducibility issue with regards to neural recommendation approaches. 

In this work, we conduct an objective study of four recently proposed state-of-the-art neural network approaches, namely NCF \cite{he2017neural}, CMN \cite{ebesu2018collaborative}, NGCF \cite{wang2019neural}, and VAE \cite{liang2018variational}, that can be used in the context of collaborative filtering, on the basis of implicit feedback, in order to address the reproducibility crisis \cite{dacrema2019we}.
It should be stated, that implicit feedback reflects users' preference through behaviours like watching videos, purchasing products, and clicking items \cite{hu2008collaborative}.
As opposed to explicit feedback, i.e. ratings and reviews, implicit feedback can be tracked automatically, but is more challenging to utilize, since only user-item interactions are collected instead of user preferences.

In Section \ref{sec:models_and_methods}, we briefly present the aforementioned approaches along with the main underlying concepts.
In Section \ref{sec:results}, we give extensive comparative results of the selected approaches on three datasets from different application domains, i.e MovieLens (movie recommendations) \cite{harper2016movielens}, Epinions (product recommendations) \cite{epinions}, and Jester (joke recommendations) \cite{jester}.
In Section \ref{sec:discussion}, we discuss the strengths and weaknesses of the selected methods based on the results, and finally, in Section \ref{sec:summary}, we summarize our work.

%%% Local Variables:
%%% mode: latex
%%% TeX-master: "../report"
%%% End:
