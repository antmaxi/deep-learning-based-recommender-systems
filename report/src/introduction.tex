\section*{Introduction}
Recommender systems are information filtering techniques that aim to predict the level of preference of a user over a specific item. In the era of big data, such techniques have attracted the interest of the scientific community, as they provide a natural approach to improving the user experience on various services, through personalization.
Classical recommender systems usually make use of either content-based or collaborative filtering approaches. Content-based filtering techniques utilize specific characteristics of an item in order to recommend additional items with similar properties, while collaborative filtering approaches utilize users' past behaviour i.e. preferences and interactions with items, as well as decisions of other users with similar interests. In most cases, collaborative filtering (CF) techniques yield improved predictions compared to the content-based approaches.
There are two main categories of methods when it comes to CF; (i) the Nearest-Neighbor techniques, and (ii) the Matrix Factorization (aka Latent Factor) methods.
As the Netflix Prize competition has demonstrated, Matrix Factorization methods are superior to classic Nearest-Neighbor techniques, as they allow the incorporation of additional information to the models, and can thus achieve improved model capacity, cf. \cite{koren2009matrix}.

\section*{Approaches to Collaborative Filtering}

Recently, both academia and industry have been in a race to design deep learning based recommender systems in an attempt to overcome the obstacles of conventional models and to achieve higher recommendation quality. In fact, deep learning can effectively capture non-linear and non-trivial user-item relationships, and also enable codification of more complex abstractions as data representations in the higher levels, cf. \cite{zhang2019deep}.
Various deep neural network architectures have been proposed and shown to be effective for predicting user preferences. Neural Collaborative Filtering (NCF) generalizes the Matrix Factorization (MF) approach by replacing the inner product utilized in MF models by a multi-layer perceptron that can learn non-linear user-item interaction functions, and thus increases the expressiveness of the MF model, cf. \cite{he2017neural}. Collaborative Memory Networks (CMN) unify the two classes of collaborative filtering models into a hybrid approach, combining the strengths of the global structure of the latent factor model, and the local neighborhood-based structure in a nonlinear fashion, by fusing a memory component and a neural attention mechanism as the neighborhood component, cf. \cite{ebesu2018collaborative}. Neural Graph Collaborative Filtering (NGCF) injects the collaborative signal into the embedding process by exploiting the user-item graph structure, so that it can effectively model high-order connectivity in the user-item interaction graph, and thus achieves improved recommendation quality, cf. \cite{wang2019neural}. Other deep learning based recommendation methods include Autoencoders, cf. \cite{sedhain2015autorec}, Variational Autoencoders, cf. \cite{liang2018variational}, and Restricted Boltzmann Machines, cf. \cite{salakhutdinov2007restricted}. 

\section*{Project's Focus}

In this project, we want to empirically study various neural network architectures that can be used in the context of collaborative filtering.
% and perform extensive comparative experiments of these architectures on various datasets from different application domains.
As authors have stated in \cite{dacrema2019we} there has been a reproducibility issue with regards to neural recommendation approaches. Therefore, we want to conduct an objective evaluation, along with hyperparameter tuning, of various architectures like the ones mentioned above, on datasets from different application domains, using standard metrics, i.e. RMSE and MAE for the cases of explicit user feedback, and Recall and NDCG (Normalized Discounted Cumulative Gain) for the cases of implicit user feedback. Finally, we also plan to combine some of these architectures in an ensemble learning context, to investigate if this further boosts the recommendation quality. The datasets we are planning to use are MovieLens, cf. \cite{harper2016movielens}, Epinions, cf. \cite{snapnets}, and Jester, cf. \cite{jester}.
%%% Local Variables:
%%% mode: latex
%%% TeX-master: "../report"
%%% End:
