% ==============================================================================
% Copyright (c) 2017 [Georg R. Pollak]  
% ==============================================================================
% ------------------------------------------------------------------------------

  % Permission is hereby granted, free of charge, to any person obtaining a copy
  % of this software and associated documentation files (the "Software"), to deal
  % in the Software without restriction, including without limitation the rights
  % to use, copy, modify, merge, publish, distribute, sublicense, and/or sell
  % copies of the Software, and to permit persons to whom the Software is
  % furnished to do so, subject to the following conditions:

  % The above copyright notice, the creator of the formuary package G.R. Pollak
  % and this permission notice shall be included in all copies or substantial portions of the Software.

% ==============================================================================
% End
% ==============================================================================
% ------------------------------------------------------------------------------

% Document class
% ------------------------------------------------------------------------------
\documentclass[
  twoColumns,
  fontsize=10pt,
  baseClass=extarticle
]{formularyETH/formularyETH}
% formuaryETH packages
% ------------------------------------------------------------------------------
\usepackage{formularyETH/formularyETH_GeneralPackages}
\usepackage{formularyETH/formularyETH_underline}
\usepackage{formularyETH/extern/formularyETH_scientific}
\usepackage{formularyETH/extern/formularyETH_tikz}
%\usepackage{formularyETH/extern/formularyETH_coding}
\usepackage{formularyETH/extern/formularyETH_algorithms}
\usepackage{formularyMacros}
% -----------------------------math Formulary----------------------------------- 
% if git@gitlab.vis.ethz.ch:formularies/math.git is used as submodule
% ------------------------------------------------------------------------------
% Uncomment next line to obtain fallback macros for math
% \usepackage{math_submodule/formularyMacros} 
% And add next line to formulary
% \input{math_submodule/math.tex}
% Other very usefull packages
% ------------------------------------------------------------------------------
\usepackage[colorinlistoftodos,prependcaption,textsize=tiny]{todonotes}
\usepackage[skip=0pt]{caption}
\usepackage{wrapfig}
\usepackage{subcaption}
\usepackage{tabularx}
% In order to use inkscape figures with transparent fillings using alpha
\usepackage{transparent}
\usepackage{blindtext}
% -----------------------------Minted------------------------------------------- 
% minted uncomment next two lines
% ------------------------------------------------------------------------------ 
% \tcbuselibrary{minted}
% \tcbset{listing engine=minted}
% ------------------------------------------------------------------------------ 
% In emacs with auctex additionally add: 
% %%% TeX-command-extra-options: "-shell-escape"
% before %%% End
% -----------------------------Minted-end--------------------------------------- 
% ==============================================================================
% Documents Definitions title, date, ...
% ==============================================================================

\title{Deep Learning based Recommender Systems}

\author{
Tselepidis Nikolaos \\
\small{\texttt{ntselepidis@student.ethz.ch}}
    \and
Goetschmann Philippe \\
\small{\texttt{pgoetsch@student.ethz.ch}}
    \and
Maksimov Anton \\
\small{\texttt{antonma@student.ethz.ch}}
    \and
Pollak Georg Richard \\
\small{\texttt{pollakg@student.ethz.ch}}
}

\date{2020 January}
  % Graphic-paths important when including pdf_tikz pictures e.g. with inkscape
  % ------------------------------------------------------------------------------ 
  \graphicspath{{figures/}
                %{figures/ch1/}
          }
% ==============================================================================
% Document begin
% 1) Introduction
% Describe your problem and state your contributions.
% 2) Models and Methods
% Describe your idea and how it was implemented to solve the problem. Survey the related work, giving credit where credit is due.
% 3) Results
% Show evidence to support your claims made in the introduction. Compare to baselines / existing work.
% 4) Discussion
% Discuss the strengths and weaknesses of your approach, based on the results. Point out the implications of your novel idea on the application concerned.
% 5) Summary
% ==============================================================================
\begin{document}

\twocolumn[{%
  \begin{@twocolumnfalse} 
  \centering
  \maketitle
% Abstract
% ======================================================================
  \begin{abstract}
    \textbf{TODO: Finish abstract}
In this work, we experimentally study four recently proposed classes of deep learning based recommender systems, namely Neural Collaborative Filtering, Collaborative Memory Networks, Neural Graph Collaborative Filtering, and Variational Autoencoders, on the basis of collaborative filtering for implicit feedback.
Our motivation came from \cite{dacrema2019we} which states that there is a reproducibility crisis in the field of neural recommendation approaches.
Hence, in this work we try to contribute to the resolution of this crisis by objectively evaluating the performance of the selected methods.
In order to be able to compare the aforementioned approaches, we modified the authors' implementations so that they can work on the same train-test splits, and also use the same evaluation metrics.
Then, we tested the selected methods on three real world datasets.
Our results show that...
%%% Local Variables:
%%% mode: latex
%%% TeX-master: "../report"
%%% End:

  \end{abstract}
  \end{@twocolumnfalse}
}]

% Introduction
% ======================================================================
\section*{Introduction}\label{sec:introduction}
\section*{Introduction}
Recommender systems are information filtering techniques that aim to predict the level of preference of a user over a specific item. In the era of big data, such techniques have attracted the interest of the scientific community, as they provide a natural approach to improving the user experience on various services, through personalization.
Classical recommender systems usually make use of either content-based or collaborative filtering approaches. Content-based filtering techniques utilize specific characteristics of an item in order to recommend additional items with similar properties, while collaborative filtering approaches utilize users' past behaviour i.e. preferences and interactions with items, as well as decisions of other users with similar interests. In most cases, collaborative filtering (CF) techniques yield improved predictions compared to the content-based approaches.
There are two main categories of methods when it comes to CF; (i) the Nearest-Neighbor techniques, and (ii) the Matrix Factorization (aka Latent Factor) methods.
As the Netflix Prize competition has demonstrated, Matrix Factorization methods are superior to classic Nearest-Neighbor techniques, as they allow the incorporation of additional information to the models, and can thus achieve improved model capacity, cf. \cite{koren2009matrix}.

\section*{Approaches to Collaborative Filtering}

Recently, both academia and industry have been in a race to design deep learning based recommender systems in an attempt to overcome the obstacles of conventional models and to achieve higher recommendation quality. In fact, deep learning can effectively capture non-linear and non-trivial user-item relationships, and also enable codification of more complex abstractions as data representations in the higher levels, cf. \cite{zhang2019deep}.
Various deep neural network architectures have been proposed and shown to be effective for predicting user preferences. Neural Collaborative Filtering (NCF) generalizes the Matrix Factorization (MF) approach by replacing the inner product utilized in MF models by a multi-layer perceptron that can learn non-linear user-item interaction functions, and thus increases the expressiveness of the MF model, cf. \cite{he2017neural}. Collaborative Memory Networks (CMN) unify the two classes of collaborative filtering models into a hybrid approach, combining the strengths of the global structure of the latent factor model, and the local neighborhood-based structure in a nonlinear fashion, by fusing a memory component and a neural attention mechanism as the neighborhood component, cf. \cite{ebesu2018collaborative}. Neural Graph Collaborative Filtering (NGCF) injects the collaborative signal into the embedding process by exploiting the user-item graph structure, so that it can effectively model high-order connectivity in the user-item interaction graph, and thus achieves improved recommendation quality, cf. \cite{wang2019neural}. Other deep learning based recommendation methods include Autoencoders, cf. \cite{sedhain2015autorec}, Variational Autoencoders, cf. \cite{liang2018variational}, and Restricted Boltzmann Machines, cf. \cite{salakhutdinov2007restricted}. 

\section*{Project's Focus}

In this project, we want to empirically study various neural network architectures that can be used in the context of collaborative filtering.
% and perform extensive comparative experiments of these architectures on various datasets from different application domains.
As authors have stated in \cite{dacrema2019we} there has been a reproducibility issue with regards to neural recommendation approaches. Therefore, we want to conduct an objective evaluation, along with hyperparameter tuning, of various architectures like the ones mentioned above, on datasets from different application domains, using standard metrics, i.e. RMSE and MAE for the cases of explicit user feedback, and Recall and NDCG (Normalized Discounted Cumulative Gain) for the cases of implicit user feedback. Finally, we also plan to combine some of these architectures in an ensemble learning context, to investigate if this further boosts the recommendation quality. The datasets we are planning to use are MovieLens, cf. \cite{harper2016movielens}, Epinions, cf. \cite{snapnets}, and Jester, cf. \cite{jester}.
%%% Local Variables:
%%% mode: latex
%%% TeX-master: "../report"
%%% End:

\section*{Models and Methods}\label{sec:models_and_methods}
% Models and Methods
% ======================================================================
\subsection*{Neural Collaborative Filtering}\label{subsec:neural_collaborative_filtering}

The Neural Collaborative Filtering (NCF) approach \cite{he2017neural} is tightly related to the Matrix Factorization (MF) method \cite{koren2009matrix}, and provides a framework that is able to express MF and also generalize it.
Instead of simply ``combining'' the user and item latent vectors using a fixed inner product, NCF utilizes a multi-layer neural network architecture that learns the interaction function from the data, and thus, increases the expressiveness of the MF model.

An instance of the NCF framework takes as input two binary one-hot encoded vectors, one for the user, and one the item, and passes them through an embedding layer, i.e. a fully connected layer that projects the sparse representations onto dense vectors.
The obtained user and item embeddings can be viewed as the user and item latent vectors, respectively.
These vectors are then fed into a multi-layer neural network architecture, the output layer of which computes a predicted score $\hat{y}_{ui}$, for the specific user-item interaction.
This score represents how likely the item i is relevant to the user u.

Considering the one-class nature of implicit feedback, the value of $y_{ui}$ is being viewed as a label, where 1 means item relevant to u, and 0 otherwise.
Therefore, training is performed by minimizing the standard binary cross-entropy loss (a.k.a. negative log loss) between $\hat{y}_{ui}$ and its target value $y_{ui}$.
It should be noted, that in order to treat the problem as a binary classification problem and use the aforementioned loss function, negative instances are also required.
These instances are uniformly sampled from the unobserved interactions in each iteration.

In the paper \cite{he2017neural}, three instantiations of the NCF approach are considered, namely, the Generalized Matrix Factorization (GMF), the Multi-Layer Perceptron (MLP), and the Neural Matrix Factorization (NeuMF).
We briefly discuss them below.

\textbf{GMF:} Considering that the output of the embedding layer is the user and item latent vectors $p_u$ and $q_i$, respectively, we can define the mapping function of the first NCF layer as the element-wise product $p_u \odot q_i$.
We can then choose the output layer to be:
\begin{equation}
    \hat{y}_{ui} = \alpha_{out}(\vec{h}^{\T}(\vec{p}_{\idxu} \odot \vec{q}_{\idxi}))\label{eq:gmf}
\end{equation}
where $a_{out}$ and $h$ denote the activation function and affine transformation weights, respectively.
Setting $a_{out}$ to the identity and enforcing $h$ to be a uniform vector of 1, we can exactly recover the MF model.
In the GMF model, the authors set $a_{out}$ to the sigmoid function and learn the values of $h$ from the data with the log loss, effectively generalizing the MF approach.

\textbf{MLP:}
Instead of computing the element-wise product of the user and item latent vector, in a neural network framework it seems definitely intuitive to concatenate them, and then feed them into a standard MLP to learn the user-item interaction function.
In this way, much more flexibility and nonlinearity can be incorporated to the model, compared to the GMF approach, and increased expressiveness can be achieved. 
In \cite{he2017neural}, the authors utilize a tower pattern for the layers, halving the layer size for each successive layer.
As activation function the use ReLU in the middle layers, and sigmoid in the output layer.

\textbf{NeuMF:}
In the NeuMF model, the authors fuse the GMF and MLP approaches into a single architecture (see Fig. \ref{fig:neumf}) in an attempt to combine the linearity of MF and non-linearity of MLP, and thus to be able to better capture complex user-item interactions.
They even add more flexibility to the fused model by allowing GMF and MLP to learn separate embeddings.
In the last layer, the outputs of GMF and MLP are concatenated and are being fed into a sigle neuron with a sigmoid activation function.

%\begin{figure}[h]
%    \centering
%    \includegraphics[width=0.8\linewidth]{images/ncf_framework.png}
%    \caption{Neural collaborative filtering framework.}
%    \label{fig:ncf}
%\end{figure}

\begin{figure}[t]
    \centering
    \includegraphics[width=0.8\linewidth]{images/neumf.png}
    \caption{Neural matrix factorization model.}
    \label{fig:neumf}
\end{figure}

%%% Local Variables:
%%% mode: latex
%%% TeX-master: "../../report"
%%% End:

\subsection*{Collaborative Memory Networks}\label{subsec:CollaborativeMemoryNetwork}
\input{src/methods/mcn.tex}
\subsection*{Neural Collaborative Filtering}\label{subsec:neural_collaborative_filtering}
\textbf{TODO: Summary by Anton}
%%% Local Variables:
%%% mode: latex
%%% TeX-master: "../../report"
%%% End:
\begin{figure}[h]
    \centering
    \includegraphics[width=0.8\linewidth]{images/ngcf.png}
    \caption{NGCF architecture.}
    \label{fig:neumf}
\end{figure}

Tha main difference of NGCF from NCF are embedding propagation layers which are 
designed to incorporate collaborative signal (especially high-order connectivities 
in users-items connections graph) into embeddings. Therefore, this is supposed to
improve the quality of recommendations.
\subsection*{Neural Collaborative Filtering}\label{subsec:neural_collaborative_filtering}
The original authors implement a variational autoencoder to solve collaborative filtering problems that are based on implicit feedback.

Variational autoencoders represent non-linear probabilistic models and can thus capture more complex relationships in the data than the linear factor models which are currently prevalent in collaborative filtering research.

They start by sampling a vector from a Gaussian distribution with $0$ mean and variance that corresponds to the number of items in the dataset. This sample is then transformed by a neural network to produce a probability vector over the items in the dataset where the probabilities should predict the users most likely next interaction. 

They use a multinomial likelihood to model the user's interaction history and empirically show that it outperforms gaussian and logistic likelihoods. 

For inference the parameters of the neural network have to be estimated and for this purpose the posterior distribution of the probability vector over items given a user and his interaction history needs to be calculated. This is not directly possible. As a solution the authors rely on variational inference which approximates the desired distribution through a fully factorized (diagonal) Gaussian distribution. Variational inference then tries to minimize the Kullback-Leiber divergence between the desired distribution and the surrogate. The new surrogate distribution grows in complexity with the number of users which might become problematic so the authors replace it's parameter by a data-dependent function (referred to as inference model) whose complexity only relies on the number of items in the data. This function introduces a new parameter $\phi$.

For learning, the log likelihood of the marginal data is lower bounded through the evidence lower bound which results in a loss function that depends on the parameters of the neural network and $\phi$. However, it's not trivially possible to take gradients with respect to $\phi$. To solve this issue the authors use the reparametrization trick which makes it possible to take the gradient with respect to $\phi$. Now it's possible to train the network with stochastic gradient descent.

To make predictions with a trained model, the user's interaction history is needed and put into a certain part of the inference model to calculate the input to the neural network which transforms it into a probability vector that predicts the probabilities with which items the user is most likely going to interact next. This means, given a new user, only two relatively efficient functions have to be called which makes predictions cheap.

%%% Local Variables:
%%% mode: latex
%%% TeX-master: "../../report"
%%% End:


% Results 
% ======================================================================
\section*{Results}\label{sec:results}
\input{src/results.tex}
% Discussion 
% ======================================================================
\section*{Discussion}\label{sec:discussion}
\textbf{TODO: }
\textit{Here we will make comments on the results that were presented in the previous section, on the advantages and limitations of the methods.}
%%% Local Variables:
%%% mode: latex
%%% TeX-master: "../report"
%%% End:

% Summary 
% ======================================================================
\section*{Summary}\label{sec:summary}
\textbf{TODO: }
\textit{Here we will summarize our work. We conducted an objective experimental study of the methods in order to contribute to the resolution of the reproducibility crisis. What were our main conclusions? Furthermore, in order to be able to objectively compare the methods some modification/standarization of the authors' codes were needed. We need to briefly discuss these things.}
%%% Local Variables:
%%% mode: latex
%%% TeX-master: "../report"
%%% End:


% ==============================================================================
% Appendix
% ==============================================================================
\setlength{\columnsep}{1mm}
\twocolumn[{%
 \centering
 \appendix{{\color{section}\LARGE\scshape\MakeUppercase\bfseries Appendix}\\[3em]}
}]
% \input{src/appendix.tex}
% ------------------------------------------------------------------------------ 
% ======================================================================
% Todo
% ======================================================================
% \input{src/General_TODO.tex}
  
% ==============================================================================
% Document end
% ==============================================================================
\bibliographystyle{plain}
\bibliography{src/citations} 
  
% ==============================================================================
% Document end
% ==============================================================================
\end{document}


%%% Local Variables:
%%% mode: latex
%%% TeX-master: t
%%% End:
